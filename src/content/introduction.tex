%----------------------------------------------------------------------------
\chapter{\bevezetes}
\label{sec:intro}
%----------------------------------------------------------------------------

A technológia fejlődésével életünk legtöbb területén egyre jobban hagyatkozunk szoftverek használatára. Általában a feladat elvégzését gyorsabban, megbízhatóbban végzik el, mintha emberi munkaerőt alkalmaznánk. Így nem meglepő, hogy olyan helyeken is alkalmazzuk, ahol a biztonságos, hibamentes üzemeltetés alapfeltétel. Az olyan rendszereket ahol a hibás működés súlyos anyagi, vagy akár ember életekben mérhető károkat okoz, biztonságkritikus rendszereknek nevezzük. Ilyen rendszerek vannak például az autóknál, vasutaknál, repülőknél, reaktoroknál, vagy egészségügyi rendszereknél. Ezeken a területeken is egyre több feladatot programok végeznek el. Mivel a helyes viselkedés létfontosságú, fontos, hogy ezeknek a szoftvereknek tudjuk bizonyítani a hibamentességét.

Viszont ezek a rendszerek általában nagyon komplexek, ezért nehéz rajtuk bizonyításokat végezni. Erre az egyik megoldás, hogy a rendszernek egy egyszerűsítésén dolgozunk, ahol csak a számunkra fontos jellemzőit tartjuk meg. Ez a modellezés. Számos modellezési technológiák vannak, én a félév során az állapottérképes modellezéssel foglalkoztam. Ezt a technikát gyakran alkalmazzák iparban is, sok eszköz elérhető ezek szerkesztésére, vizsgálatára. Hasznos tehát, hogy minél megbízhatóbb ugyanakkor elég gyors ellenőrzéseket tudjunk végezni ezeken a modelleken.

Félév során én az állapotgép modellezéssel, és az ezeken a modelleken végzett ellenőrzéssel foglalkoztam. A \hyperref[sec:archiutecture]{második fejezetben} megmutatom, hogy milyen meglévő projekteket, használtam, illetve mi módon próbáltam ezeket tovább fejleszteni.

\hyperref[sec:thetaleiras]{Harmadik fejezetben} az általam is tovább fejlesztett java állapotgép implementációt mutatom be, kiemelve, hogy mi a saját munkám.

\hyperref[sec:transzformacio]{Negyedik fejezetben} bemutatom a {\gammaSc}ek transzformációját {\thetaSc}pé.

\hyperref[sec:stateconfig]{Ötödik fejezet} az állapotgép konfigurációt mutatja meg. 

\hyperref[sec:bmc]{Hatodik fejezetben} leírom azt a modellellenőrzőt, amit a félév során sikerült implementálnom. 

\hyperref[sec:verifikacio]{Hetedik fejezet} egy rövid verifikáció az elvégzett munkámhoz. 
