%----------------------------------------------------------------------------
\chapter{\bevezetes}
\label{sec:intro}
%----------------------------------------------------------------------------

Our every day life is becoming more and more affected by softwares. When we travel our cars, planes and trains are all operating through thousands or even million lines of codes. Of course nowadays human supervision is still required, however their control over the machines decreases year by year. Softwares are not only taking control in our traveling, but other various parts of our lives, such as in health care, education, agriculture and the list is growing. The result of this take over is that our life becomes easier as softwares can solve problems more effectively.

However, these softwares are made by programmers who are humans as well. According to Steve McConnell's book Code Complete on average there are 10-50 errors in 1000 lines of code. So it is inevitable that there will be a lot of mistakes during making these huge softwares, which can consist of a couple million lines of code written by thousands of programmers.

So it is plausible that there will be bugs in our system and they are causing different problems. If the outcome of the malfunctioning software dangers great fortunes, human health or even lives than we say it is a safety critical system. It is important to make sure that these systems are error free.

There are plenty of methods for verifying our software. Static analysis is one of them. It checks the code without actually executing it, detecting possible vulnerable parts. Some problems such as simple coding errors are easy to find, however we can detect other, more complicated vulnerabilities like possible zero division or other logical errors. However checking the whole software can be impossible within a reasonable time. In this case abstracting can simplify the problem, and make it possible to analyze certain behaviors of the software.

Formal verification is a mathematical method which can not only detect possible errors, but it can prove the correctness of the program. Abstract interpretation is a formal verification method that provides efficient abstraction and iteration strategy for defining invariants in the system.

Interval abstraction is an efficient abstraction method that represents the possible program states with intervals.

In our work we develop library for interval abstraction that can be used by any abstract interpretation algorithm. We also implement some of the possible algorithms. We try to make it so it can be extended later with other abstraction types, and other algorithms.

In order to be able to verify and evaluate our work we implement our solution in Theta, an
open source verification framework (developed in our university). We can compare the performance of the different algorithms on multiple types of programs, including industrial programmable logic controller (PLC) codes from CERN, and several types of programs from the Competition on Software Verification (SV-Comp).






