%----------------------------------------------------------------------------
\chapter{Állapotgép konfiguráció}
\label{sec:stateconfig}
%----------------------------------------------------------------------------
\section{Aktív állapot}
%----------------------------------------------------------------------------
Az állapotgép a rendszerünk összes lehetséges állapotát mutatja. Analízis közben azonban fontos hogy azt is tudjuk nézni, hogy mik az aktív állapotok, azaz melyek írják le a rendszerünk pillanatnyi helyzetét.

Az aktív állapotoknak van több szabálya.
\begin{itemize}
	\item Egy régión belül, pontosan egy aktív állapot lehet (közvetlen gyerekekre nézve)
	\item Egy aktív állapot összes szülő állapota is aktív
	\item Minden régióban van aktív állapot
\end{itemize}

%----------------------------------------------------------------------------
\section{A megvalósító osztály}
%----------------------------------------------------------------------------

A StateConfiguration osztályban valósítom meg. Egy ilyen példány tartalmazza az állapotgép reprezentációt (Sc) továbbá egy listát az épp aktuális állapotokról és -egy később igen hasznosnak bizonyuló segítség- tároljuk, hogy az adott konfiguráció mely tranzíciók elsütésével érhető el a kiindulási állapotból, illetve hogy eközben milyen akciók lettek végrehajtva. Oly módon, hogy minden egyes tranzícióhoz ebben a listában két akció tartozik, egy azokhoz az akciókhoz, amik a tranzíció őrfeltétel ellenőrzés előtt hajtódnak végre, egy pedig azokhoz amik utána.

\begin{lstlisting}[language=bash,morekeywords={sudo,apt\-get},alsoletter={-},breaklines=true]
public static StateConfiguration create(final Sc sc)
\end{lstlisting}

Létrehozható egy {\thetaSc}ből.

\begin{lstlisting}[language=bash,morekeywords={sudo,apt\-get},alsoletter={-},breaklines=true]
public void init()
\end{lstlisting}

Aktiválja az állapotgépet, azaz mindenhol a kezdő állapot szerint kijelöli az aktív állapotokat

\begin{lstlisting}[language=bash,morekeywords={sudo,apt\-get},alsoletter={-},breaklines=true]
	public Collection<Transition> getFireableTransitions()
\end{lstlisting}

Visszaad egy listát az összes olyan tranzícióról, ami tüzelhető az adott konfigurációban. Vagyis a tranzíció forrásállapota aktív.

\begin{lstlisting}[language=bash,morekeywords={sudo,apt\-get},alsoletter={-},breaklines=true]
public StateConfiguration fire(final Transition tr)
\end{lstlisting}

Visszaad egy új állapotgép konfigurációt, ami a paraméterben megadott tranzíció elsütése után kialakul. Saját magát nem változtatja!

