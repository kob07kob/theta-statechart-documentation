%----------------------------------------------------------------------------
\chapter{\transzformacio}
\label{sec:transzformacio}

A Gamma keretrendszert is szeretnénk összekötni az \hyperref[sec:thetaleiras]{előző} fejezetben bemutatott {\thetaSc}pel. Ezért írtam egy segéd osztályt, ami képes a beolvasott EMF-es {\gammaSc}et átalakítani {\thetaSc}pé.
%----------------------------------------------------------------------------
\section{{\gammaSc}}
%----------------------------------------------------------------------------
Már korábban említettem az \hyperref[sec:archiutecture]{Architektúra} fejezetben, hogy a {\gammaSc} egy EMF alapú állapotgép reprezentáció. Így a sémája nem sokban különbözik az \hyperref[sec:thetaleiras]{előző} fejezetben leírt {\thetaSc}től. Ez megkönnyíti a transzformáció folyamatát. Az erre a célra írt segédosztályokat mutatom be a következő részekben


%----------------------------------------------------------------------------
\section{GammaSctoThetaSCConverter}
%----------------------------------------------------------------------------

Ez az osztály egy statikus \verb+convert()+ függvény meghívásával transzformál

\begin{lstlisting}[language=java ,breaklines=true]
public static MutableSc Convert(final StatechartDefinition gammaSC, final String name) 
\end{lstlisting}

Látható, hogy bemenetként egy beolvasott EMF gyökér elemet várunk (ez a {\gammaSc}et reprezentáló StatechartDefinition osztály) a name pedig a {\thetaSc} neve lesz.

Először beolvassuk a változókat, ezekkel létrehozok egy ExpressionConverter-t amit a következő szekcióban mutatok be. Utána beolvassuk a gyökér régiót ami az addRegiontoThetaSC() rekurzív függvény meghívásával történik utána a tranzíciókat adom az állapotgéphez (frissítve a már korábban létrehozott állapotokat)

\begin{lstlisting}[language=java ,breaklines=true]
private static void addRegiontoThetaSC(final Region reg, final MutableRegion r) 
\end{lstlisting}

Ez a függvény rekurzív, abból a szempontból, hogy ha egy régión belül van egy másik régió akkor meghívja önmagát. Minden elemhez van egy ehhez hasonló függvény, aminek a bemenete, egy gammás elem és egy annak megfelelő Thetás elem, és a transzformátor a gammás elemben lévő belső elemeket létrehozza és berakja a Thetás elembe. Ez így megy addig amíg az adott elemekben vannak újabb elemek.

%----------------------------------------------------------------------------
\section{ExpressionConverter}
%----------------------------------------------------------------------------

A legnagyobb nehézség az akciókban illetve őrfeltételekben használt kifejezések megfelelő átkonvertálása, ehhez használom a ExpressionConverter segéd osztályt. A Thetában van erre egy segítség a DispachTable. Ennek a segítségével könnyen és átláthatóan össze lehet kötni a különböző gammás Expression-öket a Thetás Expr<> osztályokkal
